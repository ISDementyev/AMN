\documentclass{article}
\usepackage{graphicx} % Required for inserting images
\usepackage{amsmath}

\title{Time propagation relations to physical constants in the Belousov–Zhabotinsky Reaction}
\author{Author: I.S.D.}
\date{Began: May 2023}

\begin{document}

\maketitle

% \section{Introduction} 

\section{Derivation}
\subsection{Basics}

The equation of any circle centered at $(0,0)$ in a 2D-Cartesian coordinate system can be expressed as the following:

\begin{equation}
    x^{2} + y^{2} = r^{2}
\end{equation}

Where $r$ is the radius of the circle. As can be deduced from empirical evidence, we can assume a linear wavefront propagation with respect to time. 
Let

\begin{equation}
    r = \nu t
\end{equation}

Where $\nu$ is the linear wavefront velocity. 
Hence, after substituting $(2)$ into $(1)$ and rearranging some variables, we can express the time progressed as a function of wave location as such:

\begin{equation}
    t = \frac{1}{\nu}\sqrt{x^{2} + y^{2}}
\end{equation}

% \subsection{Relation to Reaction Variables}

\end{document}
